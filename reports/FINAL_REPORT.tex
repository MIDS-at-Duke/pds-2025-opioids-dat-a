\documentclass[11pt,letterpaper]{article}

% Packages
\usepackage[margin=1in]{geometry}
\usepackage{graphicx}
\usepackage{booktabs}
\usepackage{hyperref}
\usepackage{xcolor}
\usepackage{titlesec}
\usepackage{fancyhdr}
\usepackage{float}
\usepackage{enumitem}
\usepackage{amsmath}

% Colors
\definecolor{headerblue}{RGB}{41, 128, 185}

% Section formatting
\titleformat{\section}{\Large\bfseries\color{headerblue}}{\thesection}{1em}{}
\titleformat{\subsection}{\large\bfseries}{\thesubsection}{1em}{}

% Header/Footer
\pagestyle{fancy}
\fancyhf{}
\fancyhead[L]{\small \textbf{TO:} State Health Policy Director}
\fancyhead[R]{\small December 2025}
\fancyfoot[C]{\thepage}
\renewcommand{\headrulewidth}{0.4pt}

% Hyperlink setup
\hypersetup{colorlinks=true, linkcolor=headerblue, urlcolor=headerblue}

\begin{document}

% -- MEMO HEADER --
\noindent
\textbf{FROM:} Tea Tafaj, Diwas Puri, Xinhao Zhang (Duke MIDS Data Science Team)\\
\textbf{RE:} Which Opioid Policies Actually Work? Evidence from Florida \& Washington\\
\textbf{DATE:} \today

\vspace{0.3cm}
\noindent\rule{\textwidth}{1pt}
\vspace{0.3cm}

% Stakeholder context box
\noindent\fbox{%
    \parbox{0.97\textwidth}{%
        \textbf{Stakeholder:} The State Health Policy Director that oversees opioid prevention strategy and budget allocation. 
        The stakeholder has strong public health expertise but limited statistical training, so results must be interpretable without technical jargon.

        \vspace{0.15cm}
        \textbf{Primary Need:} Clear evidence on whether regulatory interventions produce better outcomes than voluntary prescribing guidelines. 
        The stakeholder needs to decide how to allocate funding in the next legislative cycle—specifically, whether to expand enforcement capacity or continue investing in provider education.
        }%
}


\vspace{0.5cm}

% ==================== EXECUTIVE SUMMARY ====================
\section*{Executive Summary}

\hspace*{1.5em}\textbf{The Problem:} By 2010, prescription opioid overdoses had become a leading cause of preventable death in the United States, killing over 16,000 Americans annually. States needed evidence on which policy approaches could effectively reduce both oversupply and mortality without simply shifting users to illicit drugs.

\vspace{0.3em}

\textbf{The Question:} Do strict regulatory interventions work better than voluntary prescribing guidelines for reducing opioid harm?

\vspace{0.3em}

\textbf{The Answer:} Yes — with substantial, measurable impact.

\vspace{0.8em}

We analyzed two state interventions using county-level data (2006-2015) from the DEA's Automation of Reports and Consolidated Orders System (ARCOS), which tracks \textit{all} controlled substance shipments from manufacturers to retail pharmacies, hospitals, and practitioners:
\begin{itemize}[leftmargin=*]
    \item \textbf{Florida (2010):} Shut down pill mills, mandated Prescription Drug Monitoring Program (PDMP), deployed enforcement teams
    \item \textbf{Washington (2012):} Published prescribing guidelines, recommended dose limits (voluntary compliance)
\end{itemize}

\vspace{0.6em}

\textbf{What We Found:}

\textbf{Florida's regulatory approach produced sustained reductions.} Opioid shipments dropped by 18.1 million MME per 100,000 residents, and drug deaths fell by 2.58 per 100,000 people ($\approx$500 lives/year). Both effects are statistically significant and sustained through 2015.

\vspace{0.6em}

\textbf{Washington's voluntary approach was mixed.} Mortality declined at the time of implementation (-1.94 per 100k, p=0.007), but the post-policy trend did not differ significantly from controls (p=0.84), indicating a one-time level shift rather than a sustained effect.

\vspace{0.6em}

\textbf{Bottom Line:} Regulatory enforcement produces durable, sustained change. Voluntary education can achieve mortality reductions (likely in high-compliance systems), but fails to sustain long-term supply constraints without enforcement.

% ==================== DECISIONS TO BE MADE ====================
\section*{Decisions To Be Made}

Based on this analysis, you need to decide:

\begin{enumerate}[leftmargin=*, label=\textbf{\arabic*.}]
    \item \textbf{Should we shift budget from provider education to enforcement infrastructure?}
    
    \textit{Current allocation:} Likely heavy on Continuing Medical Education (CME) courses, guideline dissemination, awareness campaigns.
    
    \textit{Evidence suggests:} Voluntary guidelines can improve safety/mortality outcomes (likely through better clinical practice), but they fail to sustain supply reductions. Compliance erodes within 2 years without enforcement foundation. Florida-style enforcement (PDMP audit teams, clinic inspections) is required to permanently curb oversupply.
    
    \textit{Recommendation:} Reallocate at least 50\% of "education" budget to active monitoring and enforcement.
    
    \textbf{→ Next Step:} Request budget reallocation proposal by Q1 2026.

    \vspace{0.6em}
    
    \item \textbf{Will cutting supply just push users to street drugs and spike deaths?}
    
    \textit{Common fear:} Restricting prescriptions forces addicted patients to heroin/fentanyl, increasing overdoses.
    
    \textit{Evidence suggests:} Florida's data contradicts this. Deaths \textit{fell} alongside supply restrictions, suggesting the "pill mill" ecosystem was creating new addicts faster than it was serving existing ones.
    
    \textit{Recommendation:} Proceed with supply-side crackdowns, but monitor mortality closely in first 18 months.
    
    \textbf{→ Next Step:} Establish monthly mortality dashboards before implementing restrictions.

    \vspace{0.6em}
    
    \item \textbf{What's the minimum viable enforcement package?}
    
    \textit{Question:} Do we need the full Florida triad (PDMP + clinic laws + raids), or can we get 80\% of the benefit with less?
    
    \textit{Evidence suggests:} We can't isolate components from this data alone, but Washington's PDMP-only approach clearly wasn't enough.
    
    \textit{Recommendation:} Start with mandatory PDMP checks + audit authority. Add clinic licensing if initial results are weak.
    
    \textbf{→ Next Step:} Draft legislation for mandatory PDMP checks with 6-month evaluation checkpoint.
\end{enumerate}

\newpage

% ==================== BODY: FLORIDA ====================
\section*{Florida: The Regulatory Approach}

In 2010, Florida was the epicenter of the prescription opioid crisis, with "pill mills" (cash-only pain clinics) concentrated in South Florida. The state responded with a comprehensive crackdown: pain clinic registration/inspection requirements, mandatory real-time PDMP, and multi-agency law enforcement targeting high-volume prescribers.

\subsection*{What Happened to Opioid Supply?}

\begin{figure}[H]
\centering
\includegraphics[width=0.8\textwidth]{../outputs/figures/fl_prepost_did_shipments.png}
\caption{\textbf{Florida Opioid Shipments Dropped Sharply After 2010 Policy.} Red line shows Florida; blue dashed line shows 6 control states (GA, AL, SC, NC, TN, MS). Vertical line marks 2010. Immediate drop: $-18.1$M MME per 100k (p$<$0.001). Accelerating decline: $-20.0$M per year (p$<$0.001).}
\end{figure}

\vspace{0.5cm}

\noindent Florida’s shipments diverge sharply from control states immediately after 2010, consistent with a strong and sustained regulatory effect.

\vspace{1.5cm}

\subsection*{What Happened to Deaths?}

\begin{figure}[H]
\centering
\includegraphics[width=0.8\textwidth]{../outputs/figures/fl_prepost_did_mortality.png}
\caption{\textbf{Florida Drug Deaths Declined While Control States Rose.} Red line shows Florida; blue dashed shows controls. After 2010, Florida plateaus/declines while controls climb. Effect: $-2.58$ deaths per 100k (p$<$0.001) $\approx$ 500 lives saved annually.}
\end{figure}

\vspace{0.5cm}

\noindent This reduction persists through 2015, indicating the policy did not trigger a compensatory rise in illicit drug deaths.

\newpage

% ==================== BODY: WASHINGTON ====================
\section*{Washington: The Voluntary Approach}

In 2012, Washington released evidence-based "Interagency Guidelines on Prescribing Opioids for Pain" with recommendations on dosing, duration, and co-prescribing. Compliance was voluntary, promoted through CME courses and professional endorsements.

\subsection*{What Happened to Opioid Supply?}

\begin{figure}[H]
\centering
\includegraphics[width=0.8\textwidth]{../outputs/figures/wa_prepost_did_shipments.png}
\caption{\textbf{Washington's Reduction Was Temporary.} Red: Washington; blue dashed: 6 controls (OR, CO, MN, NV, CA, VA). Initial drop: −13.5M MME (p = 0.004), followed by an \textcolor{red}{upward change in trend of +6.4M MME per year} relative to the pre-policy slope ($p < 0.001$). Effect evaporated by 2015.}
\end{figure}

\vspace{0.5cm}

\noindent The brief decline quickly reversed, suggesting that voluntary guidelines did not change long-term prescribing behavior without enforcement.

\vspace{1.5cm}

\subsection*{What Happened to Deaths?}

\begin{figure}[H]
\centering
\includegraphics[width=0.8\textwidth]{../outputs/figures/wa_prepost_did_mortality.png}
\caption{\textbf{Washington Showed a One-Time Mortality Reduction.} Red: Washington; blue dashed: controls. A statistically significant drop occurred after 2012. Effect: $-1.94$ per 100k (p=0.007).}
\end{figure}

\vspace{0.5cm}

\noindent Mortality levels dropped significantly ($-1.94$ per 100k), likely capturing the impact in larger population centers where the guidelines were adopted. However, the \textit{trend} did not improve relative to controls (p=0.84), suggesting the benefit was a one-time level shift rather than a sustained trajectory change.

\newpage

% ==================== LIMITATIONS ====================
\section*{Limitations \& Caveats}

\begin{itemize}
    \item \textbf{Causation requires a strong assumption about trends.} Our analysis assumes that, in the absence of policy, Florida (or Washington) and their control states would have continued to move in parallel. The DiD design helps difference out common shocks, but Washington exhibits imperfect parallel trends in shipments relative to its control group. This divergence suggests unmeasured confounding variables (e.g., demographic shifts in age/sex, economic factors) may influence the results.
    
    \item \textbf{Anticipatory behavior in Washington.} We observe a preemptive drop in Washington's shipment volume prior to the official 2012 policy implementation. This suggests providers may have changed prescribing behavior in anticipation of the guidelines, complicating the estimation of the policy's true start date and effect size.

    \item \textbf{Substitution to illicit drugs is unmeasured.} If Florida's policy pushed users to heroin, we'd expect deaths to rise — they didn't. But we can't rule out more subtle substitution effects that our data doesn't capture.
    
    \item \textbf{Context matters.} Florida's unique "pill mill" concentration may have made regulatory approaches particularly effective. Results may differ in states with more diffuse opioid problems.
\end{itemize}

\vspace{0.3cm}

\noindent Despite these limitations, the direction and magnitude of the effects provide reliable guidance for policymaking—especially when comparing regulatory approaches with voluntary guidelines.

\vspace{0.5cm}

\noindent\textbf{Key Takeaway:} Florida's comprehensive regulatory enforcement produced immediate, sustained reductions in both supply and deaths. Washington's voluntary guidelines created temporary compliance that eroded within 2 years. For detailed hypotheses on why these approaches differed, see Appendix B.

\newpage

% ==================== APPENDICES ====================
\appendix

\section*{Appendix A: Technical Methodology}

\hspace*{1.5em}\textbf{Study Design:} Difference-in-differences (DiD) regression with county fixed effects and clustered standard errors.

\vspace{1em}

\textbf{Why DiD?} This method compares how outcomes changed in treatment states (FL, WA) versus control states before and after the policy. If treatment and control states were trending similarly before the policy ("parallel trends"), then post-policy divergence can be attributed to the intervention. This isolates the policy effect from national trends (e.g., changing fentanyl availability) that affected all states equally. However, DiD still cannot rule out all confounding; if Florida experienced other Florida-only changes at the same time as the pill-mill crackdown, those could still bias our estimates.

\vspace{1em}

\textbf{Model Specifications:}
\begin{itemize}
    \item \textbf{Model 1 (Level Change):} Tests for immediate policy impact (did the outcome jump up/down right after the policy?)
    \item \textbf{Model 2 (Trend Change):} Tests for changes in trajectory over time (did the slope of the trend change?)
\end{itemize}

\textbf{Data Sources (Complete Census, Not Sample):}
\begin{itemize}
    \item \textbf{Opioid Shipments:} U.S. Drug Enforcement Administration (DEA) Automation of Reports and Consolidated Orders System (ARCOS). This is a \textit{complete census} of all controlled substance transactions from manufacturers/distributors to retail endpoints (218.5 million transactions, 2006-2015). We filtered for retail sales only (pharmacies, hospitals, practitioners). Measured in morphine milligram equivalents (MME) per 100,000 population — a CDC-standard metric for comparing opioid potency.
    \item \textbf{Deaths:} Centers for Disease Control and Prevention (CDC) Wide-ranging Online Data for Epidemiologic Research (WONDER) vital statistics. Drug-induced mortality counts by county-year (ICD-10 codes X40-X44, X60-X64, Y10-Y14). Cells with $<$10 deaths suppressed for privacy.
    \item \textbf{Population:} U.S. Census Bureau annual county estimates (denominator for per-capita rates)
\end{itemize}

\textbf{Control States (Selected for Geographic/Demographic Similarity):}

\textbf{Florida Controls:}
\begin{itemize}
    \item \textbf{Georgia:} Similar demographics (mix of urban/rural, 13\% elderly), comparable healthcare access (Medicaid non-expansion state), similar baseline opioid environment (pre-2010 MME per capita within 15\% of FL), no major opioid regulations 2006-2015.
    \item \textbf{Alabama:} Southeastern neighbor with similar cultural attitudes toward healthcare, comparable rural healthcare challenges, parallel pre-policy opioid shipment trends, weak PDMP enforcement during study period.
    \item \textbf{South Carolina, North Carolina, Tennessee, Mississippi:} All southeastern states with similar demographic profiles, healthcare infrastructure, and absence of comprehensive opioid regulations comparable to Florida's 2010 intervention.
\end{itemize}

\textbf{Washington Controls:}
\begin{itemize}
    \item \textbf{Oregon:} Geographic neighbor with similar temperate climate, progressive healthcare policies, comparable urban/rural mix, parallel pre-2012 mortality trends, but less aggressive opioid regulation enforcement.
    \item \textbf{Colorado:} Similar demographic profile (educated, younger population), comparable healthcare access, mountainous geography with rural challenges, weaker prescription monitoring during study period.
    \item \textbf{Minnesota, Nevada, California, Virginia:} States with comparable healthcare systems and demographic characteristics, selected to provide robust control group without confounding policy interventions during 2006-2015.
\end{itemize}

\textbf{Selection Criteria:} Control states were required to have (1) no comprehensive opioid regulations implemented 2006-2015, (2) parallel pre-policy trends in outcomes (visual inspection), (3) similar demographic and healthcare characteristics, and (4) sufficient data availability (no systematic missing values).

\section*{Appendix B: Why Did Florida Work and Washington Fail?}

\textit{Note: Plausible explanations based on policy design, not proven by our analysis.}

\textbf{Hypothesis 1:} Florida had enforcement (clinic closures, prosecutions); Washington relied on voluntary compliance. Data suggests compliance erodes without consequences.

\vspace{0.3cm}

\textbf{Hypothesis 2:} Florida used multi-component approach (PDMP + oversight + enforcement); Washington used education alone.

\vspace{0.3cm}

\textbf{Hypothesis 3:} Florida's concentrated "pill mill" problem may have been uniquely suited to regulatory disruption.

\vspace{0.3cm}

\textbf{Data-Supported:} Florida's effects appeared immediately (2010-2011) and persisted through 2015.

\section*{Appendix C: Data Quality \& Preprocessing}

\subsection*{ARCOS Data Preprocessing}

Raw ARCOS data (218.5M transactions, 228GB) required extensive filtering to produce CDC-comparable opioid shipment estimates:

\textbf{1. Transaction Filtering (Removed 13.2\%):}
\begin{itemize}
    \item \textbf{Sales only:} \texttt{TRANSACTION\_CODE = 'S'} — Excluded returns, transfers, wholesale movements
    \item \textbf{Retail endpoints only:} Pharmacies, hospitals, practitioners, nurse practitioners, maintenance/detox facilities
    \item \textbf{Excluded:} Manufacturers, distributors, analytical labs, researchers
\end{itemize}

\textbf{2. Data Quality Filters (Removed 10\%):} Null counties, dosage strength/units $\le$ 0, MME $>$ 1,000,000 (outliers)

\vspace{0.3cm}

\textbf{Total removed: 47.9M transactions (22\% of raw data)}

\vspace{0.3cm}

\textbf{3. MME Calculation:} \texttt{Dosage\_Strength (mg) × DOSAGE\_UNIT (pills) × MME\_Conversion\_Factor}

\vspace{0.3cm}

\textbf{4. Aggregation:} County-year sums (10,248 obs), deduplicated 50 overlapping records

\subsection*{Validation}

Our Florida 2010 MME/capita (1,649) is 1.7-2.3x higher than CDC retail-pharmacy-only estimates (729-994). \textbf{This is expected:} CDC excludes hospitals/practitioners; we include all retail endpoints per CDC guidance for supply-side policy evaluation.

\vspace{0.3cm}

\subsection*{Handling Privacy Suppression}

CDC WONDER suppresses death counts $<$10 for privacy protection. To ensure counties had sufficient observable mortality data for trend analysis, we examined the relationship between population size and suppression frequency. A clear pattern emerged: counties below $\sim$50,000 population showed consistently high suppression rates, while larger counties rarely had missing values.

We applied two data-driven filters:
\begin{enumerate}
    \item \textbf{Population cutoff:} Median population $\geq$ 50,000
    \item \textbf{Suppression rate cutoff:} $\leq$ 40\% (suppressed in no more than 4 of 10 years)
\end{enumerate}

Counties were kept only if they met both conditions. This reduced suppressed values from 28,838 to 203 ($\sim$0.7\% remain).

For the remaining 203 suppressed observations, we imputed expected deaths using the average death rate (15.2 per 100k) $\times$ population / 100,000, constrained to integers [0, 9].

\vspace{0.3cm}

\noindent \textbf{Note on Alaska:} Per the project guidance, Alaska was excluded from the analysis due to county designation changes around 2010 (and it is not included in our treated/control state sets).

\subsection*{Final Samples}

After applying population and suppression filters:

\textbf{Florida:} 145 counties (FL + 6 controls), 1,450 county-year observations (153 imputed)

\textbf{Washington:} 91 counties (WA + 6 controls), 910 county-year observations (50 imputed)

This approach creates a cleaner, more reliable analysis panel by:
\begin{itemize}
    \item Excluding tiny counties where deaths are genuinely near-zero
    \item Keeping larger counties where suppression is meaningful
    \item Minimizing imputation (only 0.7\% of values)
\end{itemize}

\section*{Appendix D: Full Regression Results}

\begin{table}[H]
\centering
\small
\begin{tabular}{llrrrr}
\toprule
\textbf{State} & \textbf{Outcome} & \textbf{Level Change} & \textbf{p-value} & \textbf{Trend Change} & \textbf{p-value} \\
\midrule
Florida & Shipments (MME/100k) & $-18.1$M & $<$0.001*** & $-20.0$M & $<$0.001*** \\
Florida & Deaths (per 100k) & $-2.58$ & $<$0.001*** & $-0.59$ & 0.219 \\
\midrule
Washington & Shipments (MME/100k) & $-13.5$M & 0.004** & $+6.4$M & $<$0.001*** \\
Washington & Deaths (per 100k) & $-1.94$ & 0.007** & $-0.10$ & 0.843 \\
\bottomrule
\multicolumn{6}{l}{\footnotesize *p$<$0.05, **p$<$0.01, ***p$<$0.001. Standard errors clustered at county level.}
\end{tabular}
\end{table}

\section*{Appendix E: Robustness Checks}

To validate our main findings, we conducted four sensitivity analyses:

\vspace{0.3cm}

\textbf{1. Alternative Control Groups}

We re-estimated Florida's DiD using only three southeastern controls (GA, AL, SC) versus all six controls (GA, AL, SC, NC, TN, MS). Results: shipment reduction $-10.8$M (p=0.022) with 3 controls vs. $-18.1$M (p$<$0.001) with 6 controls. The effect remains significant but weaker with fewer controls, highlighting the importance of using multiple control states to improve statistical power.

\vspace{0.3cm}

\textbf{2. Population-Weighted Analysis}

Our main analysis uses county fixed effects with implicit equal weighting. We re-ran regressions weighting by county population to give more influence to larger counties. Florida's mortality effect strengthened: $-3.11$ per 100k (p$<$0.001) weighted vs. $-2.58$ (p$<$0.001) unweighted. This confirms results are not driven by small counties and, if anything, are \textit{stronger} in population-weighted specifications.

\vspace{0.3cm}

\textbf{3. Placebo Test (Pre-Policy Years)}

We tested for "effects" in 2008 (a year when no policy existed) using only 2006-2009 data. Results: MME effect p=0.006 (significant), deaths effect p=0.968 (not significant). The significant MME placebo effect suggests some pre-existing divergence in shipment trends between Florida and controls. However, the mortality placebo test passes (p=0.968), strongly supporting parallel trends for our primary outcome of interest (deaths).

\vspace{0.3cm}

\textbf{4. Exclusion of Border Counties}

To address potential spillover effects (e.g., Georgia residents crossing to Florida for prescriptions), we excluded 17 Florida counties bordering Georgia or Alabama. Results remained robust: shipments $-19.5$M (p$<$0.001), deaths $-2.51$ (p$<$0.001). The slightly stronger effects after exclusion suggest spillover is minimal and our main findings are not contaminated by cross-border dynamics.

\textbf{Conclusion:} Main findings are largely robust, with two important caveats: (1) Using only 3 control states weakens statistical power, justifying our choice of 6 controls, and (2) The shipment placebo test reveals some pre-existing divergence, suggesting caution in interpreting the shipment results as purely causal. The mortality findings remain robust across all specifications.

\section*{Appendix F: Pre-Post Summary}

This table reports simple state-year means for the treated state versus the average of its control states (pre vs. post). It is included for intuition; causal interpretation relies on the DiD results.

\begin{table}[H]
\centering
\small
\begin{tabular}{lllrcc}
\toprule
\textbf{Case} & \textbf{Group} & \textbf{Period} & \textbf{N (state-years)} & \textbf{Mean Shipments} & \textbf{Mean Death Rate} \\
 &  &  &  & \textbf{(MME/100k, millions)} & \textbf{(per 100k)} \\
\midrule
FL & Controls (avg) & Pre  & 24 & 86.2 & 12.16 \\
FL & Controls (avg) & Post & 36 & 107.2 & 13.62 \\
FL & Florida        & Pre  & 4  & 116.0 & 15.38 \\
FL & Florida        & Post & 6  & 112.4 & 14.04 \\
\midrule
WA & Controls (avg) & Pre  & 36 & 78.0 & 11.97 \\
WA & Controls (avg) & Post & 24 & 78.2 & 13.24 \\
WA & Washington     & Pre  & 6  & 97.8 & 13.68 \\
WA & Washington     & Post & 4  & 86.0 & 13.78 \\
\bottomrule
\end{tabular}
\end{table}

\vspace{1cm}

\noindent\rule{\textwidth}{0.4pt}

\vspace{0.3cm}

\noindent\textbf{Prepared by:} Tea Tafaj, Diwas Puri, Xinhao Zhang\\
\textbf{Course:} Practical Data Science (IDS 720), Duke University\\
\textbf{Code \& Data:} \url{https://github.com/MIDS-at-Duke/pds-2025-opioids-dat-a}

\end{document}
